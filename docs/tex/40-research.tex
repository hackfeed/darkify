\chapter{Исследовательская часть}

В данном разделе представлены примеры работы программного обеспечения и постановка эксперимента по сравнению результата изменения цветовой палитры изображения методом наивной инверсии и методом изменения компонентов на основе анализа цветовой карты.

\section{Пример работы программного обеспечения}

На рисунках \ref{img:start1} -- \ref{img:res3} приведены исходные состояния веб-страниц и состояния веб-страниц с измененной цветовой палитрой соответственно.

\section{Постановка эксперимента}

В данном подразделе представлены цель, описание и результаты эксперимента.

\subsection{Цель эксперимента}

Целью эксперимента является сравнение методов изменения цветовой палитры изображения. Критерием сравнения будет являться контрастность полученного изображения \cite{wcag1}.

\subsection{Описание эксперимента}

Сравнить результат преобразования цветовой палитры изображения можно при помощи сравнения полученной контрастности \cite{wcagcontrast}.

Хорошим считается отношение 4.5:1 и больше, отличным -- 7:1 и больше.

\subsection{Результат эксперимента}

В таблице \ref{tbl:experiment} представлены результаты поставленного эксперимента.

\begin{table}[H]
	\caption{Результаты сравнения методов изменения цветовой палитры изображения}
	\centering
	\resizebox{\textwidth}{!}{%
		\label{tbl:experiment}
		\begin{tabular}{|l|l|l|}
			\hline
			\textbf{\begin{tabular}[c]{@{}l@{}}Веб-страница\end{tabular}} &
			\textbf{\begin{tabular}[c]{@{}l@{}}Наивная инверсия\end{tabular}} &
			\textbf{\begin{tabular}[c]{@{}l@{}}Анализ цветовой карты\end{tabular}} \\ 
			\hline
			site1 &
			2:1 &
			7:1 \\ \hline
			site2 &
			2:1 &
			7:1 \\ \hline
			site3 &
			2:1 &
			7:1 \\ \hline
		\end{tabular}%
	}
\end{table}

\section*{Вывод}

В результате сравнения методов изменения цветовой палитры изображения было выявлено, что метод наивной инверсии может давать требуемую контрастность, но только в некоторых случаях (например, белый фон страницы и черный текст). Метод изменения цветовой палитры на основе анализа цветовой карты позволяет добиться нужной контрастности в любом случае, потому что метод предполагает собой сравнение и анализ полученных цветов изображения и выбор наиболее подходящих цветов для изменения.
