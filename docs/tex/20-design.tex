\chapter{Конструкторская часть}

В данном разделе представлены требования к программному обеспечению, схема выбранного для решения поставленной задачи алгоритма.

\section{Требования к программному обеспечению}

Программа должна предоставлять доступ к возможностям:
\begin{enumerate}[label={\arabic*)}]
	\item преобразования цветовой палитры веб-страницы из светлой в темную;
	\item сохранения визуальной структуры веб-страницыю
\end{enumerate}

К программе предъявляются требования:
\begin{itemize}
	\item размер программы не должен превышать 700 килобайт в исходном виде и 170 килобайт в сжатом \cite{jscost};
	\item программа не должна задействовать такое количество ресурсов устройства, при котором загрузка веб-страницы занимает больше 2 секунд \cite{siteload}.
\end{itemize}

\section{Разработка алгоритмов}

В данном подразделе представлены структуры данных, с которыми работает алгоритм, а также схемы алгоритма.

\subsection{Схемы алгоритма}

На рисунках \ref{img:01_A0} -- \ref{img:04_A3} представлена схема алгоритма, реализующего преобразование цветовой палитры веб-страницы.

\boximg{100mm}{01_A0}{Схема общей поставновки задачи}
\boximg{100mm}{02_A0}{Схема декомпозиции общей поставновки задачи}
\boximg{100mm}{03_A2}{Схема декомпозиции задачи составления цветовой карты}
\boximg{100mm}{04_A3}{Схема декомпозииции задачи изменения цветов на основе цветовой карты}

\subsection{DOM-дерево}

На схеме в качестве правил присутствует алгоритм изменения цвета компонентов на основе анализа цветовой карты.

Алгоритм работает с элементами DOM-дерева \cite{dom}. Пример DOM-дерева представлен на рисунке \ref{img:dom_tree}.

\img{70mm}{dom_tree}{Пример DOM-дерева}

DOM -- это представление веб-страницы в виде дерева тегов. Каждый тег дерева -- это объект. Теги являются узлами-элементами. Они образуют структуру дерева.

Элементы DOM-дерева имеют атрибуты, отвечающие за визуальное представление элемента на странице. К таким атрибутам относятся, например, фоновый цвет, цвет текста, цвет границ, ширина, высота и т.п.

Разработанный алгоритм работает с атрибутами, представляющими цветовые параметры элемента (цвет фона, текста, границ и т.п.).

Примеры атрибутов элемента DOM-дерева представлены на рисунке \ref{img:props}. В качестве элемента демонстрируется прямоугольный блок размером 120 $\times$ 70 пикселей. Для него заданы атрибуты размера (width и height), цвета фона (background), отступа от элементов внутри контейнера (padding) и параметры границы (border).

\img{40mm}{props}{Пример атрибутов элемента DOM-дерева}

\subsection{Псевдокод алгоритма}

В общем виде алгоритм изменения цвета компонентов на основе анализа цветовой карты можно описать следующим образом:

\begin{algorithm}[H]
	\caption{Изменение цвета компонентов на основе анализа цветовой карты}
	\label{alg:analchange}
	\begin{algorithmic}[1]
		\State $DomColor$ $\Leftarrow$ вычисленный доминантный цвет
		\State $DomColorLightness$ $\Leftarrow$ относительная яркость $DomColor$
		\State $NewDomColor$ $\Leftarrow$ \#121212
		\State $ColorMap$ $\Leftarrow$ составленная цветовая карта
		\ForAll {цвета из $ColorMap$}
		\State $CurColor$ $\Leftarrow$ цвет из цветовой карты
		\State $CurColorLightness$ $\Leftarrow$ относительная яркость $CurColor$
		\If {$CurColorLightness$ > $DomColorLightness$}
		\State $X$ $\Leftarrow$ контрастность $CurColor$ к $DomColor$
		\Else
		\State $X$ $\Leftarrow$ контрастность $DomColor$ к $CurColor$
		\EndIf
		\State $NewCurColor$ $\Leftarrow$ $NewDomColor$ с яркостью большей в $X$ раз
		\State $Y$ $\Leftarrow$ контрастность $NewCurColor$ к $NewDomColor$
		\While {$Y < 3$}
		\State $NewCurColor.L$ $\Leftarrow$ $NewCurColor.L$ $\times$ 1.2
		\EndWhile
		\EndFor
	\end{algorithmic}
\end{algorithm}

\section*{Вывод}

В данном разделе были представлены требования к программному обеспечению и разработаны схемы реализуемых алгоритмов.