\chapter{Конструкторская часть}

В данном разделе представлены требования к программному обеспечению, схема выбранного для решения поставленной задачи алгоритма.

\section{Требования к программному обеспечению}

Программа должна предоставлять доступ к возможностям:
\begin{enumerate}[label={\arabic*)}]
	\item преобразования цветовой палитры веб-страницы из светлой в темную;
	\item сохранения визуальной структуры веб-страницыю
\end{enumerate}

К программе предъявляются требования:
\begin{itemize}
	\item размер программы не должен превышать 700 килобайт в исходном виде и 170 килобайт в сжатом \cite{jscost};
	\item программа не должна задействовать такое количество ресурсов устройства, при котором загрузка веб-страницы занимает больше 2 секунд \cite{siteload}.
\end{itemize}

\section{Разработка алгоритмов}

На рисунках \ref{img:01_A0} -- \ref{img:04_A3} представлена схема алгоритма, реализующего преобразование цветовой палитры веб-страницы.

На схеме в качестве правил присутствует алгоритм изменения цвета компонентов на основе анализа цветовой карты.

Алгоритм работает с элементами DOM-дерева \cite{dom}. DOM -- это представление веб-страницы в виде дерева тегов. Каждый тег дерева -- это объект. Теги являются узлами-элементами. Они образуют структуру дерева. В алгоритме упоминаются элементы-родители, которые представляют собой элементы, находящиеся в DOM-дереве на один уровень выше текущего.

В общем виде алгоритм изменения цвета компонентов на основе анализа цветовой карты можно описать следующим образом:

\begin{algorithm}[H]
	\caption{Изменение цвета компонентов на основе анализа цветовой карты}
	\label{alg:analchange}
	\begin{algorithmic}[1]
		\ForAll {элементы страницы}
		\State $BaseColor$ $\Leftarrow$ цвет родительского элемента
		\State $CurColor$ $\Leftarrow$ цвет текущего элемента
		\State $X$ $\Leftarrow$ относительная яркость $CurColor$ к $BaseColor$
		\State $NewCurColor$ $\Leftarrow$ $BaseColor$ с яркостью большей в $X$ раз
		\State $Y$ $\Leftarrow$ относительная яркость $NewCurColor$ к $NewBaseColor$, где $NewBaseColor = NewCurColor$ для родительского элемента
		\If {$Y < 3$}
		\State скорректировть $NewCurColor$ в зависимости от $Y$
		\EndIf
		\EndFor
	\end{algorithmic}
\end{algorithm}

\boximg{100mm}{01_A0}{Схема общей поставновки задачи}
\boximg{100mm}{02_A0}{Схема декомпозиции общей поставновки задачи}
\boximg{100mm}{03_A2}{Схема декомпозиции задачи составления цветовой карты}
\boximg{100mm}{04_A3}{Схема декомпозииции задачи изменения цветов на основе цветовой карты}

\section*{Вывод}

В данном разделе были представлены требования к программному обеспечению и разработаны схемы реализуемых алгоритмов.