\chapter*{Введение}
\addcontentsline{toc}{chapter}{Введение}

В мире половина людей владеет смартфонами \cite{smartphones} и персональными компьютерами \cite{pcs}. Третью часть дня \cite{digitalconsumption} люди пользуются цифровыми устройствами -- смартфонами и персональными компьютерами. В этом нет ничего странного, ведь жизнь постепенно переходит в цифровое пространство: уже никого не удивить покупками \cite{onlineshopping}, просмотрами фильмов и сериалов, работой через Интернет.

В таких условиях человеческое здоровье подвергается нагрузке, в том числе при этом страдают глаза \cite{digitaleyestrain}. В течение работы за экраном компьютера или телефона, человеческий глаз испытывает перенапряжение.  Для решения этой проблемы производители программного обеспечения и разработчики веб-страниц стали добавлять в продукты функцию переключения интерфейса в темный (ночной) режим  \cite{darkmode}. Синий свет, излучаемый экранами цифровых устройств, приводит не только к перенапряжению глаз, но и к нарушению секреции мелатонина в организме человека \cite{melatonin}, что негативно влияет на сон. Использование темного режима при работе с дисплеями снимает напряжение с глаз и не нарушает секрецию мелатонина.

Помимо пользы для здоровья, темный режим положительно влияет \cite{batterysaving} на время работы от батареи устройств, оснащенных OLED \cite{OLED} дисплеями. Например, при яркости экрана в 50\%, мобильное приложение YouTube \cite{youtube} с включенным темным режимом потребляет на 15\% меньше заряда аккумулятора, чем с включенным дневным режимом \cite{batterysaving}.

Несмотря на плюсы темного режима, остаются веб-страницы, которые не поддерживают такую опцию.

Цель работы -- реализовать программное обеспечение для изменения цветовой палитры веб-страниц со светлой на темную. 

Чтобы достигнуть поставленной цели, требуется решить следующие задачи:
\begin{itemize}
	\item проанализировать цветовую карту веб-страницы, чтобы оценить какую часть цветов стоит изменить;
	\item изменить цветовую палитру веб-страницы в соответствии с полученной цветовой карты;
	\item обработать изображения, расположенные на веб-странице;
	\item обработать динамический контент (видео, анимации), расположенные на веб-странице.
\end{itemize}

Решение поставленной проблемы не очевидно, потому что помимо факторов, влияющих на здоровье пользователя, стоит учесть удобство использования преобразованными веб-страницами. Человеческий глаз лучше воспринимает черный текст на белом фоне, нежели белый текст на черном фоне, как предполагает ночной режим. Так происходит потому, что белый цвет отражает цвет каждой волны в цветовом спектре \cite{whitecolor}.

Исходя из приведенных выше фактов, стоит задача разработать такую программу, которая будет изменять цветовую палитру веб-страницы в соответствии и с требованиями к снижению нагрузки на глаза, и с требованиями относительно читаемости текста \cite{wcag}.